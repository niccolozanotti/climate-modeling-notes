\section{Foreword}\label{foreword}

\texttt{J.\ Smagorinsky}

Meteorology was one of the very first fields of physical science that
had the opportunity to exploit high speed computers for the solution of
multi-dimensional time-dependent non-linear problems. The authors of
this monograph trace the precedents from Bjerknes to von Neumann. The
numerical techniques first employed were based on a small existing body
of methodology, much of which was drawn from engineering practice, such
as the application of relaxation methods to the solution of
Poisson\textquotesingle s equation. The working repertoire of numerical
methods rapidly expanded as the physical problems grew in complexity and
as practical experience accrued. The growth was almost exclusively the
result of the innovations of the "using" physical scientists themselves.
As a consequence these advances often lacked the rigour and proof that
might have been expected from applied mathematicians. The results of
this evolution are to be found scattered throughout the meteorological
literature of the past 25 years and it became apparent that there was a
growing need for a systematic account of the rationale and development
of technique. The JOC felt that GARP\textquotesingle s needs, as
reflected by the rapid influx of new scientists into numerical
modelling, would be well served by the availability of a single
definitive source. Other related disciplines such as oceanography have
also indicated a need for a means to rapidly assimilate the accumulated
experience of meteorology. The first attempt was at the hands of two
able mathematicians, H. Kreiss and J. Oliger, who contributed a much
needed sense of mathematical unity in their monograph "Methods for the
Approximate Solution of Time-Dependent Problems" (G.P.S. No. 10, 1973).
This present volume, more specifically reflecting experience with
atmospheric models, has been written by two outstanding workers in the
field, Prof. F. Mesinger and Prof. A. Arakawa, with Dr. A. Robert as
general editor. An additional volume will be published containing
chapters on the subjects: spectral methods, global mapping problems, and
finite element methods.
