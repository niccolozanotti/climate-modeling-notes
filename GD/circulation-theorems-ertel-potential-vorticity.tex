\section{Circulation theorems and Ertel's potential vorticity}

The consideration of the turbolent motion of fluid had been going on for
sometime. The motion of vortices that showed a clear rotation character
stimulated the development of quantities that would describe the
capacity of a fluid to develop rotation with some precision. Helmholz
schubert2004 was the first to show that vorticity was conserved along
material lines in the fluid if only conservative forces were active
Thorpe2003.

Lord Kelvin introduced the concept of circulation by establishing the
following integral

\[\oint_\Omega \mathbf{v}\cdot  \mathbf{dl}\]

where the integral is along any closed curve corresponding to a material
line in the fluid. A material line is a line that maintain the
constituting particles along the movement of the flow. Integral of this
kind can be modified using Stokes theorem to result in a simple
relation between circulation and vorticity.

\[
C=\oint_\Omega \mathbf{v}\cdot  \mathbf{dl} = \int_S \nabla\times\mathbf{v}\, dS = \int_S \mathbf{\omega}\cdot \, dS
\]

where the integral is over the surface delimited by the closed circuit.
The scalar product with the oriented surfae indicates that circulation
is the average value of the vorticity component normal to the surface.
Lord Kelvin was then able to show that for a homogenous fluid the
circulation is conserved along the fluid, namely

\[\frac{D C}{Dt} = 0\]

These results were fantastic achievements for the time, but they were of
limited usefulness for the atmosphere since the atmosphere is hardly an
homogenous fluid. It is interesting to consider the issue if a suitable
generalization of Kelvin theorem can be found for compressible flows
like the atmosphere. This issue was addressed by Schutz1895 and later by
silberstein1896 that asked the question of what distribution of pressure
and density were needed to generate vorticity. It showed that it was
related to intersecting surfaces of constant pressure and density. This
fundamental result contained all the essential ingredients for the
application to geophysical flows, but Silberstein considered it as
purely mathematical problem. The merit to show that this ideas were
enormously important for the atmosphere and the ocean has to be given to
bjerknes1898 (see Thorpe2003).

To illustrate Bjerknes ideas we will consider the equations of motion
for a 3-dimensional, compressible flow on the sphere

\[\frac{\partial \mathbf{v}}{\partial t} = - (\mathbf{v}\cdot \nabla)\mathbf{v} - 2\Omega \times \mathbf{v} -\frac{1}{\rho}\nabla p - \nabla\Phi\]

where \(\Phi\) is the gravitational potential, including centrifugal
effects. The advection term can be transformed using the identity

\[(\mathbf{v}\cdot \nabla)\mathbf{v} = (\nabla \times \mathbf{v})\times\mathbf{v} + \frac{1}{2}\nabla\left( |\mathbf{v}|^2\right)\]

so

\[
\frac{\partial \mathbf{v}}{\partial t} = -\mathbf{\omega}_a\times\mathbf{v} -\frac{1}{\rho}\nabla p -\nabla \left( \Phi +\frac{1}{2}|\mathbf{v}|^2\right)
\]

where \(\mathbf{\omega} = \nabla\times\mathbf{v}\) is the relative
vorticity and \(\mathbf{\omega}_a = \mathbf{\omega} + 2\mathbf{\Omega}\)
is the total (relative plus planetary) vorticity. Taking the curl of eq
\texttt{eq6} we get\footnote{Use the vector identity
  \(\nabla \times A\times B = A(\nabla\cdot B) -B(\nabla\cdot A) + (B\cdot\nabla)A-(A\cdot\nabla)B\)
  and note that in our case the term \(B(\nabla\cdot A)\) correspond to
  \(\mathbf{v}(\nabla\cdot \nabla\times \mathbf{v})\) that is
  identically zero.}

\[\frac{\partial \mathbf{\omega}}{\partial t} = -(\mathbf{v}\cdot\nabla)\mathbf{\omega}_a +(\mathbf{\omega}_a\cdot\nabla) \mathbf{v} -\mathbf{\omega}_a \nabla\cdot\mathbf{v} -\nabla\left(\frac{1}{\rho}\right)\times\nabla p\]

Now because the planetary vorticity is independent of time we can write
it as

\[\frac{D \mathbf{\omega}_a}{Dt} = (\mathbf{\omega}_a\cdot\nabla) \mathbf{v} -\mathbf{\omega}_a \nabla\cdot\mathbf{v} -\nabla\left(\frac{1}{\rho}\right)\times\nabla p\]

Combining (\texttt{eq7}) with the continuity equation

\[\frac{D \rho}{Dt} = -\rho\nabla\cdot\mathbf{v}\]

we get

\[\frac{D \mathbf{\omega}_a}{Dt} = (\mathbf{\omega}_a\cdot\nabla) \mathbf{v}  -\nabla\left(\frac{1}{\rho}\right)\times\nabla p +\frac{\mathbf{\omega}_a}{\rho}\frac{D \rho}{Dt}\]

The presence of multiple totale derivative suggests that it is
reasonable to try to combine them, dividing by the density \(\rho\) we
can write

\[\frac{1}{\rho}\frac{D \mathbf{\omega}_a}{Dt} = \frac{1}{\rho}(\mathbf{\omega}_a\cdot\nabla) \mathbf{v}  -\frac{1}{\rho}\nabla\left(\frac{1}{\rho}\right)\times\nabla p +\frac{\mathbf{\omega}_a}{\rho^2}\frac{D \rho}{Dt}\]

or

\[\frac{D }{Dt}\left(\frac{\mathbf{\omega}_a}{\rho}\right) = \left(\frac{\mathbf{\omega}_a}{\rho }\cdot\nabla\right) \mathbf{v}  -\frac{1}{\rho}\nabla\left(\frac{1}{\rho}\right)\times\nabla p\]

Assume now that a function exist that express some property of the fluid
in almost conservative form

\[\frac{D \chi}{Dt} = S\]

where \(S\) are source and sinks terms for the property. Examining the
term, we get

\[\frac{\mathbf{\omega}_a }{\rho}\cdot \frac{D }{Dt}\nabla \chi = \left(\frac{\mathbf{\omega}_a }{\rho} \cdot\nabla\right) \frac{D  \chi}{Dt}- \left[\left(\frac{\mathbf{\omega}_a }{\rho} \cdot \nabla \right)\mathbf{v}\right]\cdot \,\nabla{ \chi}.\]

This equation can be proved by examining it component by component. if
now take the scalar product of Eq \texttt{eq7.3} with \(\nabla \chi\) we
obtain

\[\nabla\chi\cdot\frac{D }{Dt}\left(\frac{\mathbf{\omega}_a}{\rho}\right) = \left[\left(\frac{\mathbf{\omega}_a}{\rho }\cdot\nabla\right) \mathbf{v}\right]\cdot\nabla\chi  -\frac{1}{\rho}\nabla\left(\frac{1}{\rho}\right)\times\nabla p \cdot \nabla\chi\]

and then summing Eq. \texttt{eq7.4} and Eq. \texttt{eq7.5} we finally
obtain
\[\frac{\mathbf{\omega}_a }{\rho}\cdot \frac{D }{Dt}\nabla \chi +\nabla\chi\cdot\frac{D }{Dt}\left(\frac{\mathbf{\omega}_a}{\rho}\right)= \frac{\mathbf{\omega}_a }{\rho} \nabla\cdot \frac{D  \chi}{Dt}
-\frac{1}{\rho}\nabla\left(\frac{1}{\rho}\right)\times\nabla p \cdot \nabla\chi\]
or
\[
\frac{D }{Dt}\left(\frac{\mathbf{\omega}_a\cdot\nabla\chi}{\rho}\right)= \frac{\mathbf{\omega}_a }{\rho} \nabla\cdot \frac{D  \chi}{Dt}
-\frac{1}{\rho}\nabla\left(\frac{1}{\rho}\right)\times\nabla p \cdot \nabla\chi
\]

Now if the property \(\chi\) is conserved along the fluid then
\(\frac{D \chi}{Dt}=0\) and it is function only of pressure and density
(i.e. is a thermodynamic quantity), or the flow is barotropic
(\(\nabla \rho \times \nabla p =0\)) then

\[\frac{D }{Dt}\left(\frac{\mathbf{\omega}_a\cdot\nabla\chi}{\rho}\right)= 0\]

and the quantity in bracket is conserved and it is known as Ertel'ss
Potential Vorticity after Heinz Ertel Erte1942.

This quantity regulates the large scale dynamics of the atmosphere and
the ocean and it is the main guiding principle for the understanding of
the motions.

Going back to our original homogeneous fluid on the sphere we do have a
simple conserved quantity, in fact the vertical velocity \(w\) is
everywhere zero, so

\[\frac{D z}{Dt} = w = 0\]

so by setting \(\chi = z\) we also get
\(\nabla \chi = \hat{z}=\hat{k}\), i.e. the vertical unit vector.
Therefore the Ertel'ss potential vorticity reduces to

\[\mathbf{\omega}_a \cdot \nabla\chi=\mathbf{\omega}_a \cdot \hat{k} = \zeta + f\]

since \(\zeta\) is the vertical component of \(\mathbf{\omega}_a\). The
Ertel's potential vorticity so in this case is simply

\[\frac{D }{Dt}(\zeta+f)=0\]
