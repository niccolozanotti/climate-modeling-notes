\section{Flows over topography}\label{flows-over-topography}

The problem of the flows over topography is one of the most important in
geophysical fluid dynamics. The impact of mountains on the global
circulation and the development of instabilities is one of the major
phenomena. The ocean circulation features can also be impacted by the
effect of the bottom topography.

We will consider a shallow water model in a mid-latitude channel on the
\(\beta\)-plane with a bottom topography

The flow is homogeneous and therefore the density is constant
\(\rho = const\) and it will be absorbed in the other fields where
necessary, for this reasons the horizontal velocities cannot be a
function of \(z\). The longitudinal scale \(L\) is very large compared
to the vertical scale \(H_0\) so that the ratio \(H_0/L\) is very small.
This assumptions implies small vertical velocities since
\(w \approx \frac{U H_0}{L}\) so they will be smaller than horizontal
velocities.

Integrating the divergence equation from the bottom to a height \(z\),
we get

\[\int_h^z \, \frac{\partial w}{\partial z} \, dz = - \int_h^z \left( \frac{\partial u}{\partial x}+\frac{\partial v}{\partial y}\right) \, dz= - \int_h^z D \, dz\]

or

\[w(z) = w(h) - D(z-h)\]

since the horizontal velocities \((u,v)\) are independent of \(z\). The
boundary condition at the bottom requires the
\(w(h) = \frac{D h}{Dt} = \vec{v}\cdot\nabla h\), so the expression for
the vertical velocity in the fluid at the height \(z\) is

{\[w(z) =  - D(z-h)+\vec{v}\cdot\nabla h\]}

However, integrating the divergence equation over the entire depth of
the fluid

\[\int_h^{H_0} \, \frac{\partial w}{\partial z} \, dz = - \int_h^{H_0} \left( \frac{\partial u}{\partial x}+\frac{\partial v}{\partial y}\right) \, dz= - \int_h^{H_0} D \, dz\]

yields

\[w(H_0) = w(h) - D(H_0-h)\]

but the vertical velocity at the top \(w(H_0)\) is zero (rigid lid
approximation), so

\[w(h)= D (H_0-h)\]

from the definition of vertical velocity we finally have

\[w(h) =\frac{D h}{Dt}= D (H_0-h)\]

we can use this relation to eliminate the divergence \(D\) in Eq.
\texttt{wequation} to get

\[w(z)=\frac{D }{Dt}(z) =  - \frac{1}{H_0-h}\frac{D h}{Dt}(z-h)+\vec{v}\cdot\nabla h\]

or

{\[\frac{D }{Dt}(z-h) =  - \frac{1}{H_0-h}\frac{D h}{Dt}(z-h)\]}

and finally

{\[\frac{D }{Dt}\left(\frac{z-h}{H_0-h}\right) =  0\]}

this quantity is conserved, so we can use it as the quantity \(\chi\) in
the definition of the Ertel potential vorticity (Eq. \texttt{eq9}). We
neglect the horizontal component of the scalar product and so the only
conponent surviving is the vertical component

\[(\zeta+f) \frac{\partial \chi}{\partial z} = \frac{(\zeta + f)}{H_0-h}\]

to obtain the version of the potential vorticity for shallow water
systems with topography and constant density

{\[\frac{D }{Dt}\left(\frac{\zeta + f}{H_0-h}\right) =  0\]}

\subsection{Linearized vorticity equation with
topography}\label{linearized-vorticity-equation-with-topography}

We can linearize the equation considering \(h\) small with respect
\(H_0\) and assuming the relative vorticity small with respect the
planetary vorticity. The scale ratio \(\zeta/f\) can easily written in
terms of the velocity scale, \(U\), and length scale \(L\) and it is
equal to \(U/f_0L\). This adimensional ratio is an expression of the
Rossby number, requiring a small relative vorticity is so equivalent to
a small Rossby number condition.

Using the same linearization procedure as before we get

{\[\frac{\partial \zeta'}{\partial t}= -\bar{u}\frac{\partial \zeta'}{\partial x}-\gamma v'-\frac{f}{H_0}\bar{u}\frac{\partial h}{\partial x}\]}

where the topography produces the forcing term \(\mathcal{F}\) that we
have seen in eq.XXXX. The presence of the topography introduces also a
term in the zonal mean equation because of the non zero correlation with
pressure, the "form drag",

\[\frac{\partial \bar{u}}{\partial t} = \overline{v'\zeta'} -\frac{1}{H_0}\overline{p'\frac{\partial h}{\partial x}}\]

we have required a small Rossby number so the quasi geostrophic
approximation is valid and then
\(\frac{\partial p'}{\partial x} \sim f v'\),

\[-\frac{1}{H_0}\overline{p'\frac{\partial h}{\partial x}}=\frac{1}{H_0}\overline{\frac{\partial p'}{\partial x} h}=\frac{1}{H_0}\overline{v'h}\]

or

{\[\frac{\partial \bar{u}}{\partial t} = \overline{v'\zeta'} + \frac{1}{H_0}\overline{v'h}\]}

We can analyze again what are the mechanisms that contribute to the
maintenance of the zonal flow. Multiplying Eq \texttt{vorlinh} by
\(\zeta'\) after we introduce some Ekman pumping, we get

{\[\gamma \overline{\zeta'v'}=- \frac{1}{2}\frac{\partial \overline{\zeta'^2}}{\partial t}  -\epsilon\overline{\zeta'^2}-\frac{f}{H_0}\bar{u}\,\overline{\zeta'\frac{\partial h}{\partial x}}\]}

in the same way multiplying by \(h\) and zonally averaging we get

{\[\frac{\partial }{\partial t}\overline{\zeta'h} = -\bar{u}\, \overline{\frac{\partial \zeta'}{\partial x} h} -\gamma \overline{v'h} -\epsilon\overline{\zeta'h}\]}

we can use Eq \texttt{vorlinek1} and \texttt{vorlinek2} to tliminate the
term containing th triple product with the zonal mean wind and the
gradient of the topography to obtain

\[\gamma\left(\overline{v'\zeta'} + \frac{1}{H_0}\overline{v'h}\right) = -\frac{\partial }{\partial t}\left(\frac{1}{2}\overline{\zeta'^2}+ \frac{f_0}{H_0}\overline{\zeta'h}\right) -\epsilon\left( \overline{\zeta'^2}+ \frac{f_0}{H_0}\overline{\zeta'h}\right)\]

and using Eq. \texttt{ubarh}

{\[\frac{\partial \bar{u}}{\partial t}= -\frac{1}{\gamma}\frac{\partial }{\partial t}\left(\frac{1}{2}\overline{\zeta'^2}+ \frac{f_0}{H_0}\overline{\zeta'h}\right) -\frac{\epsilon}{\gamma}\left( \overline{\zeta'^2}+ \frac{f_0}{H_0}\overline{\zeta'h}\right)\]}

This expression shows that the acceleration of the zonal mean can be
divide like before in a transient and a dissipation part. Even in the
presence of topography a steady, inviscid, linear wave produce no
acceleration of the zonal mean flow.

Exercise: Show that (\texttt{ubarhfinal}) can be obtained directly ftom
the conservation of potential vorticity (\texttt{PVSW}).

\subsection{Waves forced by
topography}\label{waves-forced-by-topography}

\subsubsection{Inviscid waves}\label{inviscid-waves}

It was recognised very early in the history of meteorology that
mountains can have a strong impact on particular feature of the
circulation of the atmosphere. We will see later ( see Chapter
\texttt{chp:GeneralCirculation}) that the shape of the climatological
upper air circulation shows a direct link to the distribution of major
mountains ranges. The first analysis of the possible impact of the
mountain was carried out by Charney:1949. They started from eq.
(\texttt{vorlinh}), considering a constant basic state \(\bar{u}\) on
the \(\beta\)-plane. The mountain was supposed to be represented by a
Fourier series

\[h(x,y) = \mathcal{R} \sum_{n,m} \tilde{h}_{m,n} \sin\left(\frac{ m \pi y}{L_y}\right) \,e^{i\left(\frac{2\pi n x}{L_x}\right)}\]

or

\[h(x,y) = \mathcal{R} \sum_{n,m} \tilde{h}_{m,n} \sin(l y) \,e^{i k x}\]

and

\[l = \frac{ m \pi}{L_y} \qquad k=\frac{2\pi n}{L_x}\]

for the case of steady, linear wave we can solve eq. (\texttt{vorlinh})
assuming a similar expnasion for the streamfunction and vorticity

\[\begin{aligned}
\psi &= \sum \tilde{\psi}_{k,l}\sin(l y) \,e^{i k x} \\
\zeta &= \sum \tilde{\zeta}_{k,l}\sin(l y) \,e^{i k x}
\end{aligned}\]

then

\[\tilde{\zeta}_{k,l} = -(k^2+l^2) \tilde{\psi}_{k,l}\]

the equation then becomes

\[\bar{u}(ik) \tilde{\zeta}_{k,l} = \beta ik\tilde{\psi}_{k,l} -\frac{f_0}{H_0}\bar{u}(i k )\tilde{h}_{k,l}\]

and finally

{\[\tilde{\psi}_{k,l} = \frac{f_0}{H_0}\frac{\tilde{h}_{k,l}}{(k^2+l^2) - \beta / \bar{u}}\]}

This linear response has a resonant response for a total wavenumber

\[K_R = \sqrt{\frac{\beta}{\bar{u}}}\]

In the case of short waves \((k^2+l^2) > \beta / \bar{u}\) the dominant
balance is \(\tilde{\psi} \approx \tilde{h}\) and the \(\beta\) term is
negligible. The advection of relative vorticity is balancing the
vorticity production from the mountain:

\[\bar{u}\frac{\partial \zeta'}{\partial x} \approx -\frac{f_0}{H_0} \bar{u}\frac{\partial h}{\partial x}\]

and the response in the stremfunction is in phase with the mountain,
that is high pressure over the mountain an dlow pressure in the valley.
In the case of long waves \((k^2+l^2) < \beta / \bar{u}\) the dominant
balance is \(\tilde{\psi} \approx -\tilde{h}\) and the advection term is
negligible. The \(\beta\) term is balancing the vorticity production
from the mountain:

\[\beta \tilde{\psi} \approx -\tilde{h}\]

and the response in the streamfunction is out of phase with the
mountain, with the low pressure positioned over the mountain. A
meridional velocity is generated on the upslope of the mountain

\[\beta v \approx  -\frac{f_0}{H_0} \bar{u}\frac{\partial h}{\partial x}\]

that balance through advection of planetary vorticity the source of
negative vorticity on the mountain.

\subsubsection{Dissipation and Ekman
pumping}\label{dissipation-and-ekman-pumping}

As the flow evolve nonlinear interactions will eventually start to
activate to limit the linear evolution of the system. In rality the time
evolutio is much more complex than a simple limiting effect, but to a
first approximation is fair to consider that the main effect of
nonlinear interactions is to prevent the appearances of high amplitude
resonance. It is indeed reasonable to assume that as the amplitude grows
the small amplitude assumption of linear evolution will break down. It
is not too far out than to assume that we can roughly represent the
collective effect of nonlinear interactions as a dissipation term.
Furthermore, it is possible to show that the effect of the planetary
boundary layer result in a net dumping of vorticity at higher levels
(Ekman pumping). Taking into account these arguments we can rewrite
eq.(\texttt{vorlinh}) as

{\[\frac{\partial \zeta'}{\partial t}= -\bar{u}\frac{\partial \zeta'}{\partial x}-\gamma v'-\frac{f}{H_0}\bar{u}\frac{\partial h}{\partial x} -\epsilon\zeta'\]}

assuming a spectral expansion as in the preceding section we have

\[\bar{u}(ik) \tilde{\zeta}_{k,l} = \beta ik\tilde{\psi}_{k,l} -\frac{f_0}{H_0}\bar{u}(i k )\tilde{h}_{k,l} -\epsilon \zeta_{k,l}\]

with a solution

\[\tilde{\psi}_{k,l} =\frac{\displaystyle -\frac{f_0}{H_0}\bar{u}\tilde{h}_{k,l}}{\displaystyle \beta -(k^2+l^2)(\bar{u}-\frac{i\epsilon}{k})}\]

the solution is complex so it make sense to look at the modulus

\[|\tilde{\psi}_{k.l}|^2 = \frac{\displaystyle \left|\frac{f_0}{H_0}\bar{u}\tilde{h}_{k,l}\right|^2}{\displaystyle \left((k^2+l^2)-\frac{\beta}{\bar{u}}\right)^2 +  \left(\frac{\epsilon}{k\pi}\right)^2(k^2+l^2)^2}\]

The effect of the deviations produced by the mountain of the mean flow
can be obtained by considering eq. (\texttt{ubarhfinal}) and consider
for simplicity that there is just a single mode \((k,l)\) in the
mountain. The vorticity induced by the mountain is (see eq.
\texttt{vorsol})

{\[\tilde{\zeta}_{kl} = -\frac{K^2 f_0 \tilde{h}/H_0}{(K^2-\beta/\bar{u}) - i\kappa}, \quad K^2=(k^2+l^2) ,\quad \kappa =\frac{\epsilon}{k\bar{u}}K^2\]}

then the acceleration of the mean flow for a steady disturbance is

{\[\frac{\partial \bar{u}}{\partial t}=  -\frac{\epsilon}{\beta}\left( \overline{\zeta'^2}+ \frac{f_0}{H_0}\overline{\zeta'h}\right)\]}

where it is interesting to note that a steady eddy can indeed produce
acceleraitonb of th emean flow. Using the result from Appendix
connecting the zonal average and their spectral representation (see
Appendix \texttt{chp:spekamp}) we can write

\[\frac{\partial \bar{u}}{\partial t}=  -\frac{\epsilon}{2 \beta}\left( |\tilde{\zeta}|^2 + \frac{f_0}{H_0}\mathrm{Re}\left[ \tilde{\zeta}\tilde{h}^*\right]\right)\sin^2(l y)\]

inserting the vorticity solution (\texttt{vormont}) we obtain that

\[|\tilde{\zeta}_{kl}|^2 = -\frac{K^4 |f_0 \tilde{h}/H_0|^2}{(K^2-\beta/\bar{u})^2  + \kappa^2}\]

and

\[\mathrm{Re}\left[ \frac{f_0}{H_0}\tilde{\zeta}\tilde{h}^*\right] = -\frac{K^2 |f_0 \tilde{h}/H_0|^2 ( \beta/\bar{u}-K^2)}{(K^2-\beta/\bar{u})^2  + \kappa^2}\]

the acceleration then is

{\[\frac{\partial \bar{u}}{\partial t} = -\frac{\epsilon}{2\bar{u}} \frac{ |f_0 \tilde{h}/H_0|^2 \sin^2(l y)}{(K^2-\beta/\bar{u})^2  + \kappa^2}.\]}

The acceleration is always negative, showing that the mountains are
exercising a drag on the mean flow if it is westerly (\(\bar{u}> 0\)).
The drag vanishes if the the dissipation vanishes, but it is gets very
large as the scale of the motion is approaching the resonance
wavelength.

\subsection{Multiple equilibria}\label{multiple-equilibria}

The existence of the drag by the mountain led to the formulation of a
possible instability due to the mountain drag. The basic paradigm is
that in absence of mountain the zonal flow is always close to some
equilibrium value, where it relaxes with a certain time scale. This
simple model can be written as

\[\frac{\partial \bar{u}}{\partial t} = -\epsilon(\bar{u}-\bar{u}_E)\]

in the presence of mountain the balance is modified

{\[\frac{\partial \bar{u}}{\partial t} = -\epsilon(\bar{u}-\bar{u}_E) -D(\bar{u})\]}

where D is the drag as it is expressed by (\texttt{drag}). The precise
shape of \(D\) will depend on the particular structure of \(\bar{u}\),
in particular on the meridional structure, that will affect the waves
and in turn wil feed back on the flow, but it is clear that will exist
values for which the drag will have a strong maximum close to
\(\bar{u}_R = \beta/k^2 +l^2\).

Steady solutions will emerge if the relaxation terms balances the drag,
a condition that can be represented graphically as in Fig.
\texttt{fig:41}. The points A, B, C represents possible equilibrium
points. The C point is an equilibrium in which the zonal flow is very
close to its equilibrium value \(\bar{u}_E\), the contribution from the
waves as measured by \(D(\bar{u})\) is small and therfore the flow is
very zonal. The point \(A\) instad is a value where the zonal flow
\(\bar{u}\) is strongly affected by a large drag by the mountain,
indicating large deviations from zonal symmetry.

We can analyze the stability of the equilibria by perturbing eq
\texttt{equi} so that

\[\frac{\partial \bar{u}'}{\partial t} = -\epsilon\bar{u}' -\frac{\partial D}{\partial \bar{u}}\bar{u}' = -\left(\epsilon +\frac{\partial D}{\partial \bar{u}}\right)\bar{u}'\]

so depending on the sign of the parenthesis the point will be stable or
unstable. Inspection of the figure will show that the point at \(B\) is
unstable. This observation was first made by CharneyDevore1979 as a
theory to explain blocking.

% \begin{center}
% \includegraphics[width = .7 \textwidth]{./figs/GD/Drag.png}
% \caption{Multiple equilibria}
% \end{center}
% Image not found on the website. Check again: https://wanderer.cmcc.it/chp2chap.html